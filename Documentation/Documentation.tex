\documentclass[14pt]{article}
\usepackage{enumitem}
\usepackage{graphicx}
\begin{document}

\begin{center}
  \uppercase{Dokumentace k programu BusGrafikon}
\end{center}

\section{Uživatelská dokumentace}

\subsection{Nutné soubory}
Jízdní řády, které tato aplikace zpracovává, je nutno předložit jako soubory v poměrně specifickém formátu. Aktuální jízdní řády v tomto formátu můžete sehnat následujícím způsobem:
\begin{enumerate}
\item Na stránce portal.radekpapez.cz najděte vybranou (autobusovou) linku.
\item Vyberte možnost `Stáhnout jako xlxs` a daný soubor otevřete v programu jako např. Microsoft Excel.
\item Vyberte buňky tak, aby byl vybrán alespoň sloupeček s tarifními čísly (vč. označení `Tč`), názvy zastávek, kilometrickými údaji a jednotlivými spoji včetně záhlaví.
\item Je možno bez problémů vybrat celou tabulku - nadbytečné údaje jsou ignorovány. Nicméně tabulka by měla mít obdélníkový tvar.  \includegraphics{tabulka.png}
\item Tabulku zkopírujte do textového souboru - bude tedy vypadat jako CSV s tabulátory místo čárek.
\item Stejným způsobem si můžete upravit nebo vyrobit svůj vlastní jízdní řád.
\item Vybrané textové soubory pak zkopírujte do místa, kde leží vaše aplikace - budete se na ně odkazovat pomocí relativních cest.
\end{enumerate}

\section{Programátorská dokumentace}
299a58fae0db7e4eb6eee94a8f9dcfac HO

\end{document}